%----------------------------------------------------------------------------------------
%	SOLUTION 1
%----------------------------------------------------------------------------------------
\subsection*{Problem 1}
The energy of a photon can be quantified as:
\begin{align*}
	E &= \frac{hc}{\lambda},
\end{align*}
where, $E, h, c$ and $\lambda$ are energy, Planck's constant, speed of light and wavelength respectively. To create electron-hole pairs, the following condition needs to be satisfied:
\begin{align}\label{eq:q1_cond}
	E \geq E_{gap},
\end{align} 
where $E_{gap}$ is the band gap energy of the material in which the electron-hole pairs are required to be created. From~(\ref{eq:q1_cond}),
\begin{align}\label{eq:q1_lamb_cond}
	&\frac{hc}{\lambda} \geq E_{gap} \nonumber\\
	\implies & \lambda \leq \frac{hc}{E_{gap}}.
\end{align}
Plugging in the values of $E_{gap} = 1.42\ eV,\ h=6.63\times10^{-34}\ m^2kg/s$ and $c=3\times10^8\ m/s$ in~(\ref{eq:q1_lamb_cond}),
\begin{align*}
	\lambda &\leq \frac{6.63\times 10^{-34}\times 3\times 10^8}{1.42\times 1.6\times 10^{-19}}\ m\\
	&= 0.875\ \mu m.
\end{align*}
Therefore, a photon can have a maximum wavelength of $0.875\ \mu m$ to create electron-hole pairs in Gallium Arsenide.