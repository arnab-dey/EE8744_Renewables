%----------------------------------------------------------------------------------------
%	SOLUTION 2.i
%----------------------------------------------------------------------------------------
\subsection*{Problem 2.i}
I am not repeating the notations from the homework to conserve space. We know that complex power injection at $i^{th}$ bus is given by:
\begin{align*}
	S_i &= V_iI_i^*.
\end{align*}
Now, we know that:
\begin{align}\label{eq:q2_cur_mat}
	\begin{bmatrix}
		I\\I_{N+1}
	\end{bmatrix} &= \begin{bmatrix}
		Y & \overline{Y}\\\overline{Y}^T & y
	\end{bmatrix} \begin{bmatrix}
		V\\V_0e^{j\theta_0}
	\end{bmatrix} \nonumber\\
	&= \begin{bmatrix}
		YV + \overline{Y}V_0e^{j\theta_0}\\
		\overline{Y}^TV + yV_0e^{j\theta_0}
	\end{bmatrix}.
\end{align}
In matrix notation, therefore, we can write the following:
\begin{align*}
	S &= V \circ I^*,
\end{align*}
where $\circ$ operator denotes Hadamard product. Therefore from \ref*{eq:q2_cur_mat},
\begin{align*}
	S &= V \circ (YV + \overline{Y}V_0e^{j\theta_0})^*\\
	&= V \circ (Y^*V^* + \overline{Y}^*V_0e^{-j\theta_0})\\
	&= \text{diag}(V)(Y^*V^* + \overline{Y}^*V_0e^{-j\theta_0}),
\end{align*}
as for any two vectors $x,y \in \mathbb{R}^N$, $x \circ y = \text{diag}(x)y$.
%----------------------------------------------------------------------------------------
%	SOLUTION 2.ii
%----------------------------------------------------------------------------------------
\subsection*{Problem 2.ii}
From the previous part, we have
\begin{align*}
	S &= V \circ (Y^*V^* + \overline{Y}^*V_0e^{-j\theta_0}).
\end{align*}
If we express $V = V^{nom}+\Delta V$, and choose $V^{nom} = -Y^{-1}\overline{Y}V_0e^{j\theta_0}$, then
\begin{align*}
	&S = V \circ (Y^*V^* + \overline{Y}^*V_0e^{-j\theta_0})\\
	\implies & S^* = V^* \circ (Y^*V^* + \overline{Y}^*V_0e^{-j\theta_0})^*\\
	\implies & S^* = V^* \circ (YV + \overline{Y}V_0e^{j\theta_0})\\
	\implies & S^* = V^* \circ (Y(V^{nom}+\Delta V) + \overline{Y}V_0e^{j\theta_0})\\
	\implies & S^* = V^* \circ (YV^{nom} + Y\Delta V + \overline{Y}V_0e^{j\theta_0})\\
	\implies & S^* = V^* \circ (Y (-Y^{-1}\overline{Y}V_0e^{j\theta_0}) + Y \Delta V + \overline{Y}V_0e^{j\theta_0})\\
	\implies & S^* = V^* \circ (-\overline{Y}V_0e^{j\theta_0} + Y\Delta V + \overline{Y}V_0e^{j\theta_0})\\
	\implies & S^* = V^* \circ (Y \Delta V)\\
	\implies & S^* = (V^{nom}+\Delta V)^* \circ (Y \Delta V)\\
	\implies & S^* = (V^{nom})^* \circ Y \Delta V + \Delta V^* \circ  Y \Delta V.
\end{align*}
If we neglect second-order terms i.e. $\Delta V^* \circ  Y \Delta V$,
\begin{align*}
	S^* &= (V^{nom})^* \circ Y \Delta V \\
	&= \text{diag}((V^{nom})^*)Y \Delta V.
\end{align*}
Therefore $\Delta V$ can be solved from the above linear equation.
%----------------------------------------------------------------------------------------
%	SOLUTION 2.iii
%----------------------------------------------------------------------------------------
\subsection*{Problem 2.iii}
As given in the homework,
\begin{align*}
	K &= \text{diag}(V^{nom})Y^*\\
	J &= \begin{bmatrix}
		\text{Re}(K) & \text{Im}(K)\\\text{Im}(K) & -\text{Re}(K)
	\end{bmatrix}.
\end{align*}
From the previous part, we got,
\begin{align*}
	&S^* = (V^{nom})^* \circ Y \Delta V\\
	\implies & S  = (V^{nom}) \circ Y^* \Delta V^*\\
	\implies & S = \text{diag}(V^{nom})Y^* \Delta V^*\\
	\implies & S = K \Delta V^*\\
	\implies & P+jQ = (\text{Re}(K) + j \text{Im}(K))(\Delta V_{re}-j\Delta V_{im})\\
	\implies & P+jQ = (\text{Re}(K)\Delta V_{re} + \text{Im}(K)\Delta V_{im}) + j (\text{Im}(K)\Delta V_{re} - \text{Re}(K)\Delta V_{im}).
\end{align*}
Equating real and imaginary parts and using matrix notations, we get
\begin{align*}
	\begin{bmatrix}
		P\\Q
	\end{bmatrix} & = \begin{bmatrix}
		\text{Re}(K)\Delta V_{re} + \text{Im}(K)\Delta V_{im}\\
		\text{Im}(K)\Delta V_{re} - \text{Re}(K)\Delta V_{im}
	\end{bmatrix}\\
	&= \begin{bmatrix}
		\text{Re}(K) & \text{Im}(K)\\
		\text{Im}(K) & - \text{Re}(K)
	\end{bmatrix} \begin{bmatrix}
		\Delta V_{im}\\\Delta V_{re}
	\end{bmatrix}\\
	&= J \begin{bmatrix}
	\Delta V_{im}\\\Delta V_{re}
	\end{bmatrix}.
\end{align*}
Therefore, $[\Delta V_{im}\ \Delta V_{re}]^T$ can be solved from the following linear equations:
\begin{align*}
	\begin{bmatrix}
	\Delta V_{im}\\\Delta V_{re}
	\end{bmatrix} = J^{-1}\begin{bmatrix}
	P\\Q
	\end{bmatrix}.
\end{align*}
%----------------------------------------------------------------------------------------
%	SOLUTION 2.iv
%----------------------------------------------------------------------------------------
\subsection*{Problem 2.iv}
Let us denote $\eta = re^{j\alpha}$ and it is given that $|\eta| = r << 1$. Now,
\begin{align*}
	|1+\eta| &= |1+r\cos\alpha + j r\sin \alpha|\\
	&= \sqrt{(1+r\cos\alpha)^2+(r\sin\alpha)^2}\\
	&= \sqrt{1+2r\cos\alpha + r^2\cos^2\alpha + r^2\sin^2\alpha}.
\end{align*}
Now, $0 \leq \sin^2\alpha \leq 1$ and $0 \leq r^2 << 1$. Therefore, $0 \leq r^2\sin^2\alpha << 1$ and thus, $1+r^2\sin^2\alpha \approx 1$. Therefore,
\begin{align*}
	|1+\eta| &= \sqrt{1+2r\cos\alpha + r^2\cos^2\alpha + r^2\sin^2\alpha}\\
	&\approx \sqrt{1+2r\cos\alpha + r^2\cos^2\alpha}\\
	&= \sqrt{(1+r\cos\alpha)^2}\\
	&= 1+r\cos\alpha\\
	&= 1+\text{Re}(\eta).
\end{align*}
Now,
\begin{align*}
	\angle(1+\eta) = \arctan\left(\frac{r\sin\alpha}{1+r\cos\alpha}\right).
\end{align*}
Now, $-1 \leq \cos\alpha \leq 1$ and $0 \leq r << 1$. Therefore, $1+ r\cos\alpha \approx 1$. Similarly, $-1 \leq \sin\alpha \leq 1$, and thus $r\sin\alpha$ is also very small. Therefore,
\begin{align*}
	\angle(1+\eta) &= \arctan\left(\frac{r\sin\alpha}{1+r\cos\alpha}\right)\\
	&\approx \arctan(r\sin\alpha)\\
	& \approx r\sin\alpha \hspace*{1cm}[\text{using small angle approximation of arctan}]\\
	&= r\sin\alpha\\
	&= \text{Im}(\eta).
\end{align*}
%----------------------------------------------------------------------------------------
%	SOLUTION 2.v
%----------------------------------------------------------------------------------------
\subsection*{Problem 2.v}
It is given that,
\begin{align*}
	V &= V^{nom} + \Delta V\\
	&= V^{nom} \circ \left(\mathbbm{1} + (V^{nom})^{\circ -1}\circ \Delta V\right),
\end{align*}
where, $(V^{nom})^{\circ -1}$ is the Hadamard inverse of $V^{nom}$. Now,
\begin{align*}
	|V| &= |V^{nom}| \circ |\mathbbm{1}+(V^{nom})^{\circ -1}\circ\Delta V|.
\end{align*}
As, $|\Delta V| << \mathbbm{1}$, $|(V^{nom})^{\circ -1}\Delta V| << \mathbbm{1}$, considering $|V^{nom}| >> |\Delta V|$. Therefore, we can apply the result we obtained in part iv as follows:
\begin{align*}
	|V| &\approx |V^{nom}| \circ \left(\mathbbm{1} + \text{Re}((V^{nom})^{\circ -1}\circ\Delta V)\right)\\
	&= |V^{nom}| + |V^{nom}| \circ \text{Re}((V^{nom})^{\circ -1}\circ\Delta V)\\
	&= |V^{nom}| + |V^{nom}| \circ \text{Re} (|V^{nom}|^{\circ-1} \circ (\cos \theta^{nom}-j\sin\theta^{nom})\circ (\Delta V_{re} + j \Delta V_{im}))\\
	&= |V^{nom}| + |V^{nom}| \circ \text{Re}(|V^{nom}|^{\circ-1} \circ \cos\theta^{nom} \circ \Delta V_{re} + |V^{nom}|^{\circ-1} \circ \sin\theta^{nom}\circ \Delta V_{im}\\
	& + j |V^{nom}|^{\circ-1} \circ \cos\theta^{nom}\Delta V_{im}-j|V^{nom}|^{\circ-1} \circ \sin\theta^{nom}\circ \Delta V_{re})\\
	&= |V^{nom}| + |V^{nom}| \circ (|V^{nom}|^{\circ-1} \circ \cos\theta^{nom} \circ \Delta V_{re} + |V^{nom}|^{\circ-1} \circ \sin\theta^{nom}\circ \Delta V_{im})\\
	&= |V^{nom}| + \cos\theta^{nom} \circ \Delta V_{re} +  \sin\theta^{nom}\circ \Delta V_{im}\\
	&= |V^{nom}| + [\text{diag}(\cos\theta^{nom})\ \ \text{diag}(\sin\theta^{nom})]\begin{bmatrix}
		\Delta V_{re}\\\Delta V_{im}
	\end{bmatrix}\\
	&= |V^{nom}| + [\text{diag}(\cos\theta^{nom})\ \ \text{diag}(\sin\theta^{nom})]J^{-1}\begin{bmatrix}
	P\\Q
	\end{bmatrix}.
\end{align*}
%----------------------------------------------------------------------------------------
%	SOLUTION 2.vi
%----------------------------------------------------------------------------------------
\subsection*{Problem 2.vi}
Similarly,
\begin{align*}
	\angle V &= \theta = \angle (V^{nom} \circ \left(\mathbbm{1} + (V^{nom})^{\circ -1}\circ \Delta V\right))\\
	&= \angle V^{nom} + \angle(\left(\mathbbm{1} + (V^{nom})^{\circ -1}\circ \Delta V\right))\\
	&= \theta^{nom} + \angle(\left(\mathbbm{1} + (V^{nom})^{\circ -1}\circ \Delta V\right))\\
	&\approx \theta^{nom} + \text{Im}((V^{nom})^{\circ -1}\circ \Delta V)\\
	&= \theta^{nom} + \text{Im}(|V^{nom}|^{\circ-1} \circ (\cos \theta^{nom}-j\sin\theta^{nom})\circ (\Delta V_{re} + j \Delta V_{im}))\\
	&= \theta^{nom} + \text{Im}(|V^{nom}|^{\circ-1} \circ \cos\theta^{nom} \circ \Delta V_{re} + |V^{nom}|^{\circ-1} \circ \sin\theta^{nom}\circ \Delta V_{im}\\
	&+ j |V^{nom}|^{\circ-1} \circ \cos\theta^{nom}\Delta V_{im}-j|V^{nom}|^{\circ-1} \circ \sin\theta^{nom}\circ \Delta V_{re})\\
	&= \theta^{nom} + |V^{nom}|^{\circ-1} \circ \cos\theta^{nom}\circ\Delta V_{im}-|V^{nom}|^{\circ-1} \circ \sin\theta^{nom}\circ \Delta V_{re}\\
	&= \theta^{nom}+ |V^{nom}|^{\circ-1} \circ (\cos\theta^{nom}\circ\Delta V_{im} - \sin\theta^{nom}\circ \Delta V_{re})\\
	&= \theta^{nom} + |V^{nom}|^{\circ-1} \circ [-\text{diag}(\sin\theta^{nom})\ \ \text{diag}(\cos\theta^{nom})] \begin{bmatrix}
		\Delta V_{re}\\\Delta V_{im}
	\end{bmatrix}\\
	&= \theta^{nom} + \text{diag}(|V^{nom}|)^{-1}[-\text{diag}(\sin\theta^{nom})\ \ \text{diag}(\cos\theta^{nom})] \begin{bmatrix}
	\Delta V_{re}\\\Delta V_{im}
	\end{bmatrix}\\
	&= \theta^{nom} + \text{diag}(|V^{nom}|)^{-1}[-\text{diag}(\sin\theta^{nom})\ \ \text{diag}(\cos\theta^{nom})]J^{-1}\begin{bmatrix}
	P\\Q
	\end{bmatrix}.
\end{align*}
%----------------------------------------------------------------------------------------
%	SOLUTION 2.vii
%----------------------------------------------------------------------------------------
\subsection*{Problem 2.vii}
The code and approximation result is in the attached code (`q4\_vii\_script.m').