%----------------------------------------------------------------------------------------
%	SOLUTION 1
%----------------------------------------------------------------------------------------
\subsection*{Problem 1.i}
The swing equations and simplified turbine-governor model are given by:
\begin{align}\label{eq:q1_swing}
	\dot{\delta}_g &= \omega_g-\omega_s \nonumber\\
	M_g\dot{\omega}_g &= P_g^m-P_g^e \nonumber \\
	\tau_g\dot{P}_g^m &= P_g^r-P_g^m - \frac{1}{R_g\omega_s}(\omega_g-\omega_s).
\end{align}
The generator reference power input is given by:
\begin{align}\label{eq:q1_gen_ref}
	P_g^r &= P_g^*+\alpha_g\left(\xi - \sum_{j\in \mathcal{G}}P_j^*\right),
\end{align}
where $\alpha_g$ is the AGC participation factor and $\sum_{g \in \mathcal{G}}\alpha_g=1$. $\xi$ is the AGC state whose evolution is given by:
\begin{align*}
	\dot{\xi} &= -\xi-\text{ACE}+\sum_{g\in \mathcal{G}}P_g^e\\
	\text{ACE} &= b\left((1/G)\sum_{g \in \mathcal{G}}(w_g) - w_s\right),
\end{align*}
and $b > 0$ and there are a total of $G$ generators in the area. Economic dispatch problem is formulated as follows:
\begin{align}
	&\text{min}_{P_g, g \in \mathcal{G}} \sum_{g\in \mathcal{G}}C_g(P_g) \nonumber\\
	&\text{s.t. } \sum_{g \in \mathcal{G}}P_g = P_{load} + P_{loss}(P_{\mathcal{G}}).
\end{align}
It is given in the question that after a load change in the system, the new load is given by $\overline{P}_{load} = P_{load}+\Delta P_{load}$ and let us denote the corresponding changes in the loss as $\overline{P}_{loss} = P_{loss}+\Delta P_{loss}$. In steady state, after the load change, we get the following:
\begin{align}\label{eq:q1_ss}
	M_g \dot{\overline{\omega}}_g = 0 & \implies \overline{P}^m_g = \overline{P}^e_g \nonumber\\
	\tau_g \dot{\overline{P}}_g^m = 0 & \implies \overline{P}^r_g - \overline{P}^m_g - \frac{1}{R_g \omega_s}(\overline{\omega}_g-\omega_s) \nonumber\\
	\dot{\overline{\xi}} = 0 & \implies \overline{\xi} = -\frac{b}{G}(\overline{\omega}_g - \omega_s) + \sum_{g \in \mathcal{G}}\overline{P}_g^e.
\end{align}
From \ref{eq:q1_gen_ref}, summing over all generators in the area, we get,
\begin{align*}
	\sum_{j \in \mathcal{G}} \overline{P}_j^r &= \sum_{j \in \mathcal{G}}P_j^* + \sum_{j \in \mathcal{G}} (\alpha_g \overline{\xi}) - \sum_{j\in \mathcal{G}} \alpha_g \sum_{j\in \mathcal{G}}P_j^*\\
	\sum_{j \in \mathcal{G}} \overline{P}_j^r &= \overline{\xi},
\end{align*}
as $\sum_{j \in \mathcal{G}}\alpha_j = 1$. Therefore,
\begin{align*}
	&\sum_{j \in \mathcal{G}} \overline{P}_j^r = -\frac{b}{G}(\overline{\omega}_g-\omega_s)+\sum_{j \in \mathcal{G}}\overline{P}_g^e\\
	\implies & \sum_{j \in \mathcal{G}}\overline{P}_j^r - \sum_{j \in \mathcal{G}}\overline{P}_g^e = -\frac{b}{G}(\overline{\omega}_g-\omega_s)\\
	\implies & \sum_{j \in \mathcal{G}}\overline{P}_j^r - \sum_{j \in \mathcal{G}}\overline{P}_g^m = -\frac{b}{G}(\overline{\omega}_g-\omega_s)\\
	\implies & \frac{1}{R_g \omega_s}(\overline{\omega}_g-\omega_s) + \frac{b}{G}(\overline{\omega}_g-\omega_s) = 0\\
	\implies & \overline{\omega}_g = \omega_s.
\end{align*}
Therefore, in steady-state, $\overline{\text{ACE}} = 0$. Thus,
\begin{align*}
	\overline{\xi} &= \sum_{j \in \mathcal{G}} \overline{P}_g^e = \overline{P}_{load}+\overline{P}_{loss} = P_{load}+\Delta P_{load} + P_{loss}+\Delta P_{loss}.
\end{align*}
Now, as $P_g^*$ is the outcome of economic dispatch optimization problem, it has to satisfy the constraint, $\sum_{j \in \mathcal{G}}P_g^* = P_{load}+P_{loss}$. Hence, from \ref*{eq:q1_ss},
\begin{align*}
	\overline{P}_g^m &= \overline{P}_g^r\\
	&= P_{g}^*+\alpha_g \left(\overline{\xi}-\sum_{j \in \mathcal{G}}P_j^*\right)\\
	&= P_{g}^*+\alpha_g \left(P_{load}+\Delta P_{load} + P_{loss}+\Delta P_{loss} - P_{load}-P_{loss}\right)\\
	&= P_{g}^*+\alpha_g \left(\Delta P_{load} + \Delta P_{loss}\right).
\end{align*}
%----------------------------------------------------------------------------------------
%	SOLUTION 1.ii
%----------------------------------------------------------------------------------------
\subsection*{Problem 1.ii}
From \ref{eq:q1_ss},
\begin{align*}
	\overline{P}_g^e &= \overline{P}_g^m\\
	&= P_{g}^*+\alpha_g \left(\Delta P_{load} + \Delta P_{loss}\right).
\end{align*}
%----------------------------------------------------------------------------------------
%	SOLUTION 1.iii
%----------------------------------------------------------------------------------------
\subsection*{Problem 1.iii}
Using Lagrange multiplier $\lambda$ and KKT conditions for the given economic dispatch optimization problem, we get the dual objective as follows:
\begin{align*}
	\left(\sum_{g \in \mathcal{G}}C_g(P_g)\right) + \lambda \left(P_{load}+P_{loss}(P_{\mathcal{G}}) - \sum_{g\in \mathcal{G}}P_g\right)
\end{align*}
Denoting the optimal variables as $P_g^*$ and $\lambda^*$, we get,
\begin{align*}
	\frac{\partial}{\partial P_g} \left(\sum_{g \in \mathcal{G}}C_g(P_g)\right)+\frac{\partial}{\partial P_g}\lambda \left(P_{load}+P_{loss}(P_{\mathcal{G}}) - \sum_{g\in \mathcal{G}}P_g\right) &= 0.
\end{align*}
Carrying out the derivative, we get:
\begin{align}\label{eq:q1_kkt}
	& C^{\prime}_g(P_g^*) - \lambda^* \left(1-\frac{\partial}{\partial P_g}P_{loss}(P_{\mathcal{G}^*})\right) = 0 \nonumber\\
	\implies & C^{\prime}_g(P_g^*) - \frac{\lambda^*}{\Lambda_g^*} = 0,
\end{align}
where, $\Lambda_g^* = \left(1-\frac{\partial}{\partial P_g}P_{loss}(P_{\mathcal{G}^*})\right)^{-1}$.
%----------------------------------------------------------------------------------------
%	SOLUTION 1.iv
%----------------------------------------------------------------------------------------
\subsection*{Problem 1.iv}
AGC participation factor is given by:
\begin{align*}
	\alpha_g &= \frac{(C^{\prime\prime}_g(P_g^*))^{-1}}{\sum_{j \in \mathcal{G}}(C^{\prime\prime}_j(P_j^*))^{-1}}.
\end{align*} 
It is given that,
\begin{align}\label{eq:q1_cond}
	\overline{\Lambda}_g^*C^{\prime}_g(\overline{P}_g^*)-\Lambda_g^*C^{\prime}_g(P_g^*) = (\overline{P}_g^*-P_g^*)C^{\prime\prime}_g(P_g^*).
\end{align}
Now, from \ref{eq:q1_kkt} and \ref{eq:q1_cond}, we get,
\begin{align*}
	&\overline{\lambda}^* - \lambda^* = (\overline{P}_g^*-P_g^*)C^{\prime\prime}_g(P_g^*)\\
	\implies & \frac{\overline{\lambda}^* - \lambda^*}{C^{\prime\prime}_g(P_g^*)} = (\overline{P}_g^*-P_g^*).
\end{align*}
where $\overline{\lambda}^*$ is the optimal value of Lagrange multiplier of the optimization problem solution after the load change. Summing over all the generators, we get:
\begin{align*}
	(\overline{\lambda}^*-\lambda^*)\sum_{j \in \mathcal{G}}(C^{\prime \prime}_j(P_j^*))^{-1} = \sum_{j \in \mathcal{G}}(\overline{P}_j^*-P_j^*).
\end{align*}
From the derivation in part (ii),
\begin{align*}
	\overline{P}_g^e &= P_{g}^*+\alpha_g \left(\Delta P_{load} + \Delta P_{loss}\right)\\
	&= P_g^* + \alpha_g\left(\sum_{j \in \mathcal{G}}\overline{P}_j^* - \sum_{j \in \mathcal{G}}P_j^*\right)\\
	&= P_g^* + \frac{(C^{\prime\prime}_g(P_g^*))^{-1}}{\sum_{j \in \mathcal{G}}(C^{\prime\prime}_j(P_j^*))^{-1}}\left(\sum_{j \in \mathcal{G}}\overline{P}_j^* - \sum_{j \in \mathcal{G}}P_j^*\right)\\
	&= P_g^* + \frac{(C^{\prime\prime}_g(P_g^*))^{-1}}{\sum_{j \in \mathcal{G}}(C^{\prime\prime}_j(P_j^*))^{-1}} \sum_{j \in \mathcal{G}}(\overline{P}_j^*-P_j^*)\\
	&= P_g^* + \frac{(C^{\prime\prime}_g(P_g^*))^{-1}}{\sum_{j \in \mathcal{G}}(C^{\prime\prime}_j(P_j^*))^{-1}} (\overline{\lambda}^*-\lambda^*)\sum_{j \in \mathcal{G}}(C^{\prime \prime}_j(P_j^*))^{-1}\\
	&= P_g^* + (C^{\prime\prime}_g(P_g^*))^{-1} (\overline{\lambda}^*-\lambda^*)\\
	&= P_g^* + (C^{\prime\prime}_g(P_g^*))^{-1}(\overline{P}_g^*-P_g^*)C^{\prime\prime}_g(P_g^*)\\
	&= P_g^* + \overline{P}_g^*-P_g^*\\
	&= \overline{P}_g^*.
\end{align*}
Therefore, under the given conditions [A1]-[A2], the setpoints as derived from economic dispatch solution becomes exactly equal to th electrical output power of the generators. This helps to minimize the impact of inaccurate loss penalty factor consideration.
%----------------------------------------------------------------------------------------
%	SOLUTION 1.v
%----------------------------------------------------------------------------------------
\subsection*{Problem 1.v}
Convex cost function with convex feasible set would satisfy [A1]. In lossless networks, [A2] would get automatically satisfied.